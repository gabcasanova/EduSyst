\documentclass[main.tex]{subfiles}

\begin{document}

% PORTUGUÊS --------------------------------------------------------------
\begin{titlepage}
% Título
\begin{center}
    \large
    \textbf{\MakeUppercase{Resumo}}
\end{center}

% Texto
Este trabalho tem como objetivo o desenvolvimento e a implementação do EduSyst, um sistema projetado para enfrentar os desafios na gestão acadêmica das escolas públicas de ensino médio no estado do Rio de Janeiro. O objetivo do estudo é criar uma plataforma que possibilite a informatização das escolas, simplifique a organização de dados e melhore a comunicação entre alunos, professores e responsáveis. O EduSyst visa fornecer uma solução técnica abrangente e intuitiva para a administração escolar, oferecendo funcionalidades para a visualização e gestão de informações acadêmicas. O trabalho aborda a concepção e implementação de uma plataforma que promove a eficiência e a transparência na gestão escolar, alinhando-se aos objetivos de modernização e melhoria na administração das instituições de ensino médio.
\\
\\
Palavras-chave: gestão acadêmica, administração escolar, plataforma educacional, eficiência, modernização.
\end{titlepage}
%--------------------------------------------------------------------------

%INGLÊS -------------------------------------------------------------------
\begin{titlepage}
% Título
\begin{center}
    \large
    \textbf{\MakeUppercase{Abstract}}
\end{center}

% Texto
This work aims to develop and implement EduSyst, a system designed to address the challenges in academic management of public high schools in the state of Rio de Janeiro. The objective of the study is to create a platform that enables the digitalization of schools, simplifies data organization, and enhances communication between students, teachers, and guardians. EduSyst seeks to provide a comprehensive and intuitive technical solution for school administration, offering functionalities for the visualization and management of academic information. The work addresses the design and implementation of a platform that promotes efficiency and transparency in school management, aligning with the goals of modernization and improvement in the administration of high school institutions.
\\
\\
Keywords: academic management, school administration, educational platform, efficiency, modernization.
\end{titlepage}
%--------------------------------------------------------------------------

\end{document}

